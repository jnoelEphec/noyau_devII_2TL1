\documentclass[12pt]{article}
\usepackage[utf8]{inputenc}
\usepackage[T1]{fontenc}
\usepackage[french]{babel}
\usepackage[margin=3.5cm]{geometry}

% Use Sans Serif font
\renewcommand{\familydefault}{\sfdefault}

\title{
    {\Huge Cahier des charges}
    \\ \vspace{1em} \textbf{\textsc{Groupe 2TL1-5}}
    \vspace{1em}
}
\author{
    Aurelle Awountsa
    \and Mark Borca
    \and Morgane Leclerc
    \and Thomas Namurois
    \and François Temmerman
    \vspace{1em}
}
\date{\today}

\begin{document}

\maketitle

\section{Présentation du client}

L’EPHEC, est une haute école créée par Serge Gasquard en 1969.
Ils possèdent à l’heure actuelle 4 implantations situées à 
Schaerbeek, Louvain-la-Neuve, Auderghem et Woluwe-Saint-Lambert.
L’établissement propose une vingtaine de bacheliers et formations.

\section{Présentation du projet}

Au vu de la situation sanitaire actuelle,
l’EPHEC désire une solution permettant aux élèves et aux professeurs de communiquer entre eux à distance.
L’EPHEC nous charge donc de développer un programme permettant la communication écrite
, avec une ou plusieurs personnes,
au travers d’une interface.

\section{Objectif client}

L’objectif du client
, par la réalisation de ce projet,
est l’amélioration de la communication en distanciel
au sein des différents établissements scolaires de l’EPHEC.

\section{Intervenants}

% Using minipage to avoid the list being cut by a pagebreak
\begin{minipage}{\linewidth}
\begin{itemize}
    \item Les clients: la heute-école EPHEC
    \item Le prestataire: le groupe \texttt{2TL1-5}
    \item La supervision des prestataires se fera par Mr Noël, professeur du groupe \texttt{2TL1-5} à Louvain-la-Neuve
\end{itemize}
\end{minipage}

\section{Public cible \& utilisateurs}

\begin{minipage}{\linewidth}

Nos publics cibles sont:

\begin{itemize}
    \item Les étudiants de l'EPHEC
    \item Les professeurs travaillant à l'EPHEC
    \item Les différents membres du personnel de l'EPHEC
\end{itemize}
\end{minipage}

\section{Demandes fonctionnelles}

\begin{enumerate}
    \item L’utilisateur doit avoir la possibilité d’envoyer un message à destination d’un groupe ou d’un autre utilisateur
    \item L’utilisateur doit avoir la possibilité de sélectionner le groupe ou l’utilisateur à qui il veut envoyer un message
    \item L’utilisateur doit avoir la possibilité de recevoir le message d’un autre groupe ou utilisateur
    \item Les messages échangés entre les utilisateurs doivent avoir un affichage permettant leur lecture
    \item L’historique complet des messages doit être accessible
    \item L’utilisateur doit voir les groupes et utilisateurs avec lesquels il a une discussion active
    \item L’utilisateur doit avoir la possibilité créer des groupes
    \item L’utilisateur doit avoir la possibilité quitter des groupes
    \item L’utilisateur doit voir les autres utilisateurs membres d’un groupe dont il est membre
    \item Les conversations doivent être affichées en fonction de l’utilisateur identifié
\end{enumerate}

\section{Contraintes}

\begin{minipage}{\linewidth}
\begin{enumerate}
    \item L’application devra être développée en Python
    \item MongoDB fera office de base de données
    \item L’interface graphique devra être faite avec la librairie Kivy 2.0
    \item Nous devrons utiliser un dépôt Git pour la gestion de version, dépôt hébergé 
sur Github
\end{enumerate}
\end{minipage}

\section{Charte graphique \& ergonomique}

L’interface doit être simple, efficace,
permettre aux utilisateurs d’utiliser le programme sans difficulté
afin qu’il puisse être utilisable par tous les membres de l’EPHEC.

\section{Enveloppe budgétaire}

La demande émanant des professeurs de l’EPHEC
dans le cadre d’un cours pour le groupe prestataire
et ne demandant pas de dépense (serveur, matériel informatique,...),
aucune enveloppe budgétaire et rémunération ne sera fournie.

\section{Planification}

Ce projet doit être développé pour la fin décembre. Chaque semaine, le lundis 
matins à 8h30, une rencontre avec le client, en accord avec ce dernier, sera prévue 
pour présenter l’état d’avancement et faire une mise au point sur les besoins du 
client. 

\begin{minipage}{\linewidth}

Pour l’état d’avancement, certaines dates ont déjà été définies : 

\begin{itemize}
    \item \textbf{18/10}: Définition du Produit Minimum Viable
    \item \textbf{25/10}: Définition du cahier des charges
    \item \textbf{08/11}: Démonstration du Produit Minimum Viable  
    \item \textbf{22/11}: Démonstration de la solution complète 
    \item \textbf{29/11}: Démonstration de la fiabilité de la solution par la réalisation de différents tests
\end{itemize}
\end{minipage}


\end{document}

